\documentclass[a4paper,12pt]{report}

\usepackage[utf8]{inputenc}
\usepackage[russian]{babel}
\inputencoding{utf8}
\usepackage{graphicx}
\graphicspath{{images/}}

\makeatletter
\renewcommand\chapter{\par%
\thispagestyle{plain}%
\global\@topnum\z@
\@afterindentfalse
\secdef\@chapter\@schapter}
\makeatother

\usepackage{geometry} % Меняем поля страницы
\geometry{left=2cm}% левое поле
\geometry{right=1.5cm}% правое поле
\geometry{top=3cm}% верхнее поле
\geometry{bottom=2cm}% нижнее поле

\begin{document}

\begin{titlepage}
\begin{center}
\vspace{8cm}
Санкт-Петербургский Государственный Университет. \\ Математико-Механический факультет. \\ Кафедра астрономии. \\
\vspace{2cm}
Мазурин К.Э. \\
\vspace{1cm}
\Large
{\bf ‘’Исследование алгоритмов эфемеридного обеспечения навигационных спутников.’’\\ Курсовая работа, 3-й курс. } \\
\vspace{1cm}
Научный руководитель -- Петров Сергей Дмитриевич.\\
\vspace{13cm}
Санкт-Петербург -- 2015 год.
\end{center}
\end{titlepage}

{\large\tableofcontents}
\newpage

\large
\chapter*{Введение}
\addcontentsline{toc}{chapter}{Введение}
\par ГНСС (Глобальные Навигационные Системы Станций) появились уже более 20 лет назад. Приблизительно в одно время в России (ГЛОНАСС) и в США (GPS) была начата работа над запуском навигационных
спутников. И на данный момент это единственные функционирующие системы. В данный момент они обе широко используются в гражданских целях, при том, что изначально предназначались исключительно для 
военных. В сферу их применения входит автомобильная, морская и авиационная навигация, сопровождение астрометрических наблюдений, отслеживание движения литосферных плит и многое другое. Для
большинства из этих целей требуется максимально высокая точность измерений. И в данный момент это является одной из главных проблем, стоящих перед учеными. Работа по улучшению точности ведется 
непрерывно, совершенствуются устройства, улучшаются алгоритмы. На данный момент удалось достичь сантиметровой точности измерения координат\par
\bigskip
Основным принципом работы ГНСС является отправка сигналов (навигационных сообщений) со спутников и прием этих сигналов специальными пользовательскими устройствами (приемниками). В сообщениях 
содержится эфемеридная информация о спутнике, испустившем сигнал (содержание нав.сообщения в GPS и ГЛОНАСС различное). Затем сигналы обрабатываются, и пользователь получает информацию о своих 
координатах. Для этого приемнику требуется одновременно видеть минимум четыре спутника. До каждого из них определяется расстояние, т.н. псевдодальность, и, предварительно вычислив
координаты спутника, приемник ищет точку пересечения сфер с центрами в точках, где находятся спутники, и радиусами, равными псевдодальностям. Четыре спутника требуются для того, чтобы помимо трех
координат наблюдателя вычислять еще и поправку часов приемника. Для определения скорости есть два метода: использование эффекта Доплера и измерение скорости изменения координат. В первом случае
наблюдаются непредвиденные ошибки и потеря точности, так что в ответственных измерениях всегда предпочитают второй способ. \par
\bigskip
Принципиальным различием в системах GPS и ГЛОНАСС является содержание навигационного сообщения и последующая обработка сигналов приемниками, а именно определение координат спутников. GPS для этого 
использует шесть кеплеровых элементов орбиты: большую полуось, эксцентриситет, наклон орбиты, аргумент перигея, долготу восходящего узла и среднюю аномалию. Именно эти элементы, а также поправки к 
ним, передаются в навигационном сообщении. Затем определение координат спутника происходит при помощи простого аналитического алгоритма. \par
\bigskip
При определении координат навигационных спутников ГЛОНАСС используется алгоритм численного интегрирования системы дифференциальных уравнений движения спутника, а именно метод Рунге-Кутты четвертого
порядка. В качестве начальных данных подаются эфемериды спутника на какой-то известный момент времени. Навигационное сообщение спутников ГЛОНАСС состоит из эфемерид спутника, меток времени и другой 
информации.

\chapter*{Постановка задачи}
\addcontentsline{toc}{chapter}{Постановка задачи}
\par Целью данной курсовой работы являлось модифицирование алгоритма расчета эфемерид спутников ГЛОНАСС, на основе соответствующего алгоритма, который используется в системе GPS. Требовалось заменить
численное интегрирование на более простой метод, использующий кеплеровы элементы орбиты. Если бы результаты работы исходного и модифицированного алгоритмов совпали в пределах 
заданной точности, можно было бы говорить, что полученный метод является достойной альтернативой базовому.

\chapter*{Проделанная работа}
\addcontentsline{toc}{chapter}{Проделанная работа}
В качестве языка программирования мною был выбран Python, поскольку он является наиболее интуитивно-понятным высокоуровневым языком. Изначальная направленность на читаемость кода позволяет даже
незнакомым с этим языком людям быстро понимать, как работает программа. \par 
В первую очередь мною был реализован алгоритм расчета положения наблюдателя по известным координатам четырех спутников. То есть, как описано выше, найти точку пересечения четырех сфер с известными
центрами и радиусами. Это сводится к решению системы из четырех (или более) нелинейных уравнений. Данная задача описана в книге \cite{kaplan06}. После этого была написана 
программа, определяющая координаты спутников GPS по времени на момент его наблюдения, а также полученным из навигационного сообщения элементам орбиты и поправкам к ним. Это классическая задача 
небесной механики, заключающаяся в последовательном нахождении ряда величин, определяющих положение спутника на орбите (например, радиусвектора и аргумента широты).  Алгоритм, использованный в этой 
программе, приводится в книге 
\cite{prakt11}. Эти две программы в совокупности позволяют определять координаты наблюдателя по данным GPS спутников. При проверке правильности их работы результат совпал с ожидаемым с точностью до 
нескольких сотых секунд дуги по широте и долготе, что соответствует дециметровой точности. Затем был запрограммирован алгоритм пересчета эфемерид навигационного спутника ГЛОНАСС на текущий момент
времени, представленный в \cite[p.~56]{ikd}. Для проверки правильности его работы из интернета был загружен RINEX-файл с эфемеридами спутников ГЛОНАСС. Полученные результаты совпали с истинными с 
точностью до 100 метров. Такая низкая точность была вызвана тем, что в алгоритме учитывался всего лишь один член разложения геопотенциала Земли в ряд по сферическим функциям. Помимо этого промежуток 
интегрирования системы был взят равным 30 минутам, при том, что в \cite[p.~56]{ikd} рекомендовано, чтобы данная величина не превышала 15 минут. \par
Далее началась само модифицирование. Для начала требовалось определить параметры орбиты спутника. Существует множество способов определения орбиты; мною был выбран метод, использующий для этого
положение и скорость спутника в заданный момент времени. Данный алгоритм подробно разобран в книге \cite[p.~155]{kholsh}. Результатом работы этого алгоритма на примере спутника Р1 стали 6 параметров
орбиты:
\bigskip
\begin{center}
Спутник №1, 01.01.2015, 00.15\\

Большая полуось \textbf{a} =  25509.6057743\\
Эксцентриситет \textbf{e} =  0.00048608172471\\
Наклон орбиты \textbf{i} =  64.195247593\\
Долгота восходящего узла \textbf{OMEGA} =  169.243384632\\
Аргумент перигея \textbf{omega} =  25.7106478818\\
Средняя аномалия \textbf{M} =  -14.1208312337\\
\end{center}
\bigskip
\par Данные величины достаточно точно совпали с теми, что были найдены мной в интернете. Следующим шагом была адаптация алгоритма определения координат спутника по элементам орбиты. Требовалось
отказаться от учета поправок, так как в данном случае используются мгновенные параметры орбиты, а не средние, как в исходном. Финальным шагом моей работы являлся совместный запуск двух программ
и сравнение результатов работы модифицированного алгоритма и алгоритма из \cite{ikd}. Результаты представлены в таблице:
\bigskip
\begin{center}
\begin{tabular}{|c|c|c|}
\hline
& Исходный алгоритм & Модифицированный алгоритм \\ \hline
X & -10000.6037458 & -10000.6253246 \\ \hline
Y & 20909.5051194 & 20909.3601445 \\ \hline
Z & 10625.1015309 & 10625.15882 \\ \hline
\end{tabular}
\end{center}

\chapter*{Заключение}
\addcontentsline{toc}{chapter}{Заключение}
\par Мы видим, что в пределах заданной точности ($\approx100$ метров) результаты работы алгоритмов совпадают (величины на рисунке приведены в километрах). Так что можно 
говорить о том, что полученный метод пригоден для поставленной задачи. Он призван заменить громоздкое интегрирование системы из 6 дифференциальных уравнений простыми аналитическими выкладками, тем 
самым упрощаются вычисления и уменьшается вероятность появления нежелательных ошибок.

\newpage
\addcontentsline{toc}{chapter}{Литература}
\begin{thebibliography}{9}

\bibitem{kaplan06}
	E. Kaplan, C. Hegarty,
	\emph{Understanding GPS: principles and applications}.
	Artech House, Massachusetts,
	2nd edition,
	2006.

\bibitem{ikd}
	Интерфейсный Контрольный Документ ГЛОНАСС,
	редакция 5.1,
	Москва,
	2008.

\bibitem{leick04}
	A. Leick,
	\emph{GPS satellite surveying}.
	John Wiley \& sons, New Jersey,
	3d edition,
	2004.

\bibitem{kholsh}
	К.В. Холшевников, В.Б. Титов,
	\emph{Задача двух тел. Учебное пособие}.
	Санкт-Петербургский государственный университет, Санкт-Петербург,
	2007

\bibitem{prakt11}
	\emph{Небесные и земные координаты. Учебное пособие по астрометрической практике}.
	Санкт-Петербургский государственный университет, Санкт-Петербург,
	2011

\end{thebibliography}

\end{document}
