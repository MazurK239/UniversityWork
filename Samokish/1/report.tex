\documentclass[a4paper,12pt]{report}

\usepackage[utf8]{inputenc}
\usepackage[russian]{babel}
\usepackage{amsmath}
\usepackage{multicol}
\setlength{\columnsep}{1cm}
\inputencoding{utf8}

\usepackage{geometry} % Меняем поля страницы
\geometry{left=2cm}% левое поле
\geometry{right=1.5cm}% правое поле
\geometry{top=3cm}% верхнее поле
\geometry{bottom=2cm}% нижнее поле

\begin{document}

\begin{center}
\Large
{\bf Решение диференциального уравнения второго порядка \\ методом разностной прогонки (методом сеток).}
\end{center}

\underline{Постановка задачи:} требуется численно решить линейное неоднородное дифференциальное уравнение второго порядка
\begin{equation}
y''(x) + p(x)y'(x) + q(x)y(x) = f(x)
\end{equation}
с граничными условиями III типа:
\begin{equation}
y'(a) = \alpha y(a) + A ; 
y'(b) = \beta y(b) + B
\end{equation}

\underline{Используемые формулы:} 
\begin{itemize}
\item{Приближенное вычисление производных:
\begin{equation}
y'(a) = \frac{y(a+h) - y(a)}{h} + R
\end{equation}
\begin{equation}
y'(a) = \frac{y(a) - y(a-h)}{h} + R
\end{equation}
\begin{equation}
y'(a) = \frac{y(a+h) - y(a-h)}{2h} + R
\end{equation}
\begin{equation}
y''(a) = \frac{y(a+h) - 2y(a) + y(a-h)}{h^2} + R
\end{equation}
}
\item{Сетка интергирования имеет вид: $x_k = a + h * k$, $k = 0, \ldots, n$; $h = \frac{b-a}{n}$, где $n$ - число узлов интегрирования, а $h$ - шаг сетки.}
\item{Численное решение будем искать в виде: $y_k \approx y(x_k)$}
\item{Система уравнений будет иметь вид:
\begin{equation}
\left \{ \begin{aligned} 
& -b_{0}y_0 + c_{0}y_1 = d_0 \\ 
a_{k}y_{k-1} & - b_{k}y_k + c_{k}y_{k-1} = d_k \\ 
a_{n}y_{n-1} & - b_{n}y_n = d_n 
\end{aligned} \right.
\end{equation}
Где:
\begin{equation}
\begin{aligned}
{\bf b_0} & = \frac{1}{h} + \frac{p_0}{2} - \frac{h}{2}q_0 + \alpha & {\bf c_0} & = \frac{1}{h} + \frac{p_0}{2} & {\bf d_0} & = A + \frac{h}{2}f_0 \\
{\bf a_k} & = \frac{1}{h^2} - \frac{p_k}{2h} & {\bf b_k} & = \frac{2}{h} - q_k & {\bf c_k} & = \frac{1}{h} + \frac{p_k}{2h} & {\bf d_k} & = f_k \\
{\bf a_n} & = \frac{p_n}{2} - \frac{1}{h} & {\bf b_n} & = \frac{p_n}{2} - \frac{1}{h} + \frac{h}{2}q_n + \beta & {\bf d_n} & = B - \frac{h}{2}f_n
\end{aligned}
\end{equation}
}
\item{Для решения системы уравнений использовались формулы:
\begin{gather}
y_{k-1} = \alpha_{k-1}y_k + \beta_{k-1} \\
\alpha_k = \frac{c_k}{b_k - a_{k}\alpha_{k-1}} \\
\beta_k = \frac{a_{k}\beta_{k-1} - d_k}{b_k - a_{k}\alpha_{k-1}}
\end{gather}
}
\end{itemize}

\newpage
\begin{center}
\Large
{\bf Написанная программа (Python):}
\end{center}

\begin{multicols}{2}
[
\underline{\bf Главная программа:} \\
]
\noindent
{\bf import} math \\
{\bf import} sys \\
{\bf import} three\_diag {\bf as} td \\
\\
aa = {\bf float}(sys.argv[1]) \\
bb = {\bf float}(sys.argv[2]) \\
n = {\bf int}(sys.argv[3]) \\
output\_f\_name = sys.argv[4] \\
out\_file = {\bf open}(output\_f\_name, 'w') \\
\\
{\bf def} p(x): \\
\hspace*{2em}    {\bf return} x + 1 / (1 + x) \\
{\bf def} q(x): \\
\hspace*{2em}    {\bf return} math.sqrt(1 + x**2) \\
{\bf def} f(x): \\
\hspace*{2em}    {\bf return} 1 - x**2 \\
\\
alpha = 0 \\
A = 0 \\
beta = -2 \\
B = 1 \\
h = (bb - aa) / n \\
a = [] ;
b = [] ;
c = [] ;
d = [] \\
a.append(0) \\
b.append(1/h + p(aa)/2 - q(aa)*h/2 + alpha) \\
c.append(1 / h + p(aa) / 2) \\
d.append(A + f(aa) * h / 2) \\
\\
{\bf for} k {\bf in range}(1, n): \\
\hspace*{2em}    x\_k = aa + h * k \\
\hspace*{2em}    a.append(1 / h**2 - p(x\_k) / (2 * h)) \\
\hspace*{2em}    b.append(2 / h**2 - q(x\_k)) \\
\hspace*{2em}    c.append(1 / h**2 + p(x\_k) / (2 * h)) \\
\hspace*{2em}    d.append(f(x\_k)) \\
\\
a.append(p(bb) / 2 - 1 / h) \\
b.append(p(bb)/2 - 1/h + q(bb)*h/2 + beta) \\
c.append(0) \\
d.append(B - f(bb) * h / 2) \\
\\
y, alphak, betak = td.solution(a, b, c, d) \\
\end{multicols}

\bigskip

\underline{\bf Модуль three\_diag, решающий систему уравнений:} \\
\\
\noindent
{\bf def} solution(a, b, c, d): \\
\hspace*{2em}    n = len(b) - 1 \\
\hspace*{2em}    y = [0] \\
\hspace*{2em}    alpha = [] \\
\hspace*{2em}    beta = [] \\
\hspace*{2em}    alpha.append(c[0] / b[0]) \\
\hspace*{2em}    beta.append(-d[0] / b[0]) \\
\hspace*{2em}    {\bf for} k {\bf in} {\bf range}(1, n + 1): \\
\hspace*{4em}    alpha.append(c[k] / (b[k] - a[k] * alpha[k-1])) \\
\hspace*{4em}    beta.append((a[k] * beta[k-1] - d[k]) / (b[k] - a[k] * alpha[k-1])) \\
\hspace*{4em}    y.append(0) \\
\\
\hspace*{2em}    y[n] = beta[n] \\
\hspace*{2em}    {\bf for} k {\bf in} {\bf range}(n-1, -1, -1): \\
\hspace*{4em}         y[k] = alpha[k] * y[k+1] + beta[k] \\
 \\
\hspace*{2em}    {\bf return} y, alpha, beta\\

\newpage

\begin{center}
{\bf Результаты работы программы для сетки с $n = 10, 20, 40$ узлами. \\ Указаны только точки, соответствующие друг другу.} \\
\bigskip
\begin{tabular}{|c|c|c|c|}
\hline
& n = 10 & n = 20 & n = 40 \\ \hline
$y_0$ & 0.360613 & 0.358790 & 0.358334 \\ \hline
$y_1$ & 0.363658 & 0.361870 & 0.361423 \\ \hline
$y_2$ & 0.372355 & 0.370593 & 0.370152 \\ \hline
$y_3$ & 0.385716 & 0.383967 & 0.383529 \\ \hline
$y_4$ & 0.402536 & 0.400789 & 0.400352 \\ \hline
$y_5$ & 0.421431 & 0.419676 & 0.419237 \\ \hline
$y_6$ & 0.440877 & 0.439107 & 0.438665 \\ \hline
$y_7$ & 0.459265 & 0.457475 & 0.457028 \\ \hline
$y_8$ & 0.474952 & 0.473142 & 0.472690 \\ \hline
$y_9$ & 0.486322 & 0.484497 & 0.484041 \\ \hline
$y_{10}$ & 0.491845 & 0.490013 & 0.489556 \\ \hline
\end{tabular}
\end{center}

\end{document}

