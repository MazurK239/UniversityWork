\documentclass[a4paper,12pt]{report}

\usepackage[utf8]{inputenc}
\usepackage[russian]{babel}
\usepackage{amsmath}
\usepackage{multicol}
\usepackage{amssymb}
\setlength{\columnsep}{1cm}
\inputencoding{utf8}

\usepackage{geometry} % Меняем поля страницы
\geometry{left=2cm}% левое поле
\geometry{right=1.5cm}% правое поле
\geometry{top=3cm}% верхнее поле
\geometry{bottom=2cm}% нижнее поле

\begin{document}

\begin{center}
\Large
{\bf Решение интегрального уравнения II рода методом Ритца.}
\end{center}

\underline{Постановка задачи:} требуется численно решить интергальное уравнение II рода
\begin{equation}
u(x) - \int_a^b K(x,t)u(t)dt = f(x)
\end{equation}

\underline{В данной задаче:}
\begin{equation}
\begin{aligned}
& K(x,t) = ln(1+0.35xt) \\
& f(x) = ln(1+x) \\
& [a,b] = [0,1]
\end{aligned}
\end{equation}

В первую очередь стоит отметить, что данный метод применим только для положительно-определенных симметричных операторов $K(x,t)$. Данная задача удовлетворяет этому условию.
Для решения необходимо найти функцию $u$, минимизирующую энергетический функционал $\mathcal{F}(u)  = (Au,u) - 2(f,u)$, где 
$Au = u - \int_a^b K(x,t)u(t)dt; \; (g,y) = \int_a^b g(t)y(t)dt$ - скалярное произведение. \\
Введем набор линейно-независимых координатных функций $\left \{ \phi_k \right \}_1^n$ и представим решение исходного уравнения в виде линейной комбинации этих функций
\begin{equation}
u_n = \sum_{k=1}^n c_k \phi_k
\end{equation}
Воспользовавшись условием минимизации энергетического функционала, после некоторых преоразований получим систему уравнений
\begin{equation}
\sum_{k=1}^n c_k (A\phi_k, \phi_i) = (f, \phi_i) \quad i = 1 \dots n
\end{equation}
из которой можно найти коэффициенты $c_k$. \newline

Для вычисления интегралов используется квадратурная формула Гаусса:
\begin{equation}
\int_a^b f(x)dx \approx \sum_{i=1}^n A_i f(\xi_i)
\end{equation}
Узлы квадратурной формулы $\xi_i$ - корни полинома Лежандра степени $n$, которые можно найти методом итераций:
\begin{equation}
\xi_i^{(0)} = cos(\pi \frac{4i-1}{4n+2}); \quad \xi_i^{(k+1)} = \xi_i^{(k)} - \frac{P_n(\xi_i^{(k)})}{P_n^\prime(\xi_i^{(k)})}
\end{equation}
Где $P_n^\prime(x) = \frac{n}{1-x^2}(P_{n-1}(x) - xP_n(x))$ - производная от полинома Лежандра.
Веса квадратурной формулы вычисляются по формуле: $A_i = \frac{2}{(1-\xi_i^2)(P_n^\prime(\xi_i))^2}$ \newline

\end{document}
