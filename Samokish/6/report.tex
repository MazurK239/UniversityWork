\documentclass[a4paper,12pt]{report}

\usepackage[utf8]{inputenc}
\usepackage[russian]{babel}
\usepackage{amsmath}
\usepackage{multicol}
\usepackage{amssymb}
\setlength{\columnsep}{1cm}
\inputencoding{utf8}

\usepackage{geometry} % Меняем поля страницы
\geometry{left=2cm}% левое поле
\geometry{right=1.5cm}% правое поле
\geometry{top=3cm}% верхнее поле
\geometry{bottom=2cm}% нижнее поле

\begin{document}

\begin{center}
\Large
{\bf Решение одномерного параболического уравнения методом сеток. (Простейшая явная и неявная схемы)}
\end{center}

\underline{Постановка задачи:} требуется численно решить одномерное уравнение теплопроводности
\begin{equation}
\frac{\partial u}{\partial t} = a_{0}(x,t)\frac{\partial^2 u}{\partial x^2} + a_1(x,t)\frac{\partial u}{\partial x} + a_2(x,t)u + f(x,t)
\end{equation}
с граничными условиями:
\begin{equation}
\frac{\partial u}{\partial x}(a,t) = \psi_0 ; \quad
\frac{\partial u}{\partial x}(b,t) = \psi_1
\end{equation}
и начальным условием: 
\begin{equation}
u(x,0) = \phi(x)
\end{equation}

\underline{В данной задаче:}
\begin{equation}
\begin{aligned}
&a_{0} = 1 \quad a_1  = 0 \quad a2  = -1 \quad f = 0 \\
&\phi(x) = \frac{1}{(1+x^2)^2} \quad \psi_0 = \psi_1 = 0
\end{aligned}
\end{equation}

Сетка имеет вид: $x_i = ih$, $t_k = k\tau$, $i=0 \dots n; \; k=0 \dots M$ где $h=\frac{1}{n}; \; \tau = \frac{1}{2M}$ . 
\\Область: $x\in[0,1]; \quad t\in[0,\frac{1}{2}]$
\\Численное решение будем искать в виде: $u_i^k \approx u(x_i, t_k)$

\underline{Составляем сеточные уравнения:} 
\begin{equation}
\mathcal{L}_h u_i^k = a_{0}\frac{u_{i+1}^k-2u_i^k+u_{i-1}^k}{h^2} + a_1\frac{u_{i+1}^k-u_{i-1}^k}{h} + a_2 u_i^k
\end{equation}

Для граничных условий уравнения будут выглядеть так:
\begin{equation}
\frac{-u_2^k + 4u_1^k - 3u_0^k}{2h} = \psi_0(t_k)
\end{equation}
\begin{equation}
\frac{u_{n-2}^k - 4u_{n-1}^k + 3u_n^k}{2h} = \psi_1(t_k)
\end{equation}

\begin{itemize}
\item{{\bf Простейшая явная схема:}\\
\begin{equation}
\frac{u_i^{k+1} - u_i^k}{\tau} = \mathcal{L}_h u_i^k + f(x_i,t_k)
\end{equation}
Разрешаем уравнения относительно $u_i^{k+1}$:
\begin{equation}
u_i^{k+1}= A_i^ku_{i-1}^k + B_i^k u_i^k + C_i^k u_i^{k+1} + D_i^k
\end{equation}
Где:
\begin{equation}
\begin{aligned}
A_i^k & = \sigma a_{0} - \frac{h}{2}\sigma a_1 & B_i^k & = 1 - 2\sigma a_{0} + \tau a_2 & \sigma = \frac{\tau}{h^2} \\
C_i^k & = \sigma a_{0} + \frac{h}{2}\sigma a_1 & D_i^k & = \tau f(x_i,t_k)
\end{aligned}
\end{equation}
}
\item{{\bf Простейшая неявная схема:}
\begin{equation}
\frac{u_i^{k} - u_i^{k-1}}{\tau} = \mathcal{L}_h u_i^k + f(x_i,t_k)
\end{equation}
Прямой счет невозможен, три незвестных:
\begin{equation}
A_i^k u_{i-1}^k - B_i^k u_i^k + C_i^k u_{i+1}^{k} = D_i^k
\end{equation}
Где:
\begin{equation}
\begin{aligned}
A_i^k & = \sigma a_{0} - \frac{h}{2}\sigma a_1 & B_i^k & = 1 + 2\sigma a_{0} - \tau a_2 & \sigma = \frac{\tau}{h^2} \\
C_i^k & = \sigma a_{0} + \frac{h}{2}\sigma a_1 & D_i^k & = -\tau f(x_i,t_k) - u_i^{k-1}
\end{aligned}
\end{equation}
На каждом шаге по $t$, дополняя систему уравнений граничными условиями, решаем ее методом матричной прогонки.
}
\end{itemize}

\end{document}
